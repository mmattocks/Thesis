PROLEGOMENON

\section{Purpose and Use of Prolegomenon Chapter}
I have chosen to include an introductory chapter to this thesis in order to clarify the reasoning of the arguments presented in the data chapters. As even the literature reviews introducing the data chapters are structured according to the methodological considerations outlined here, reference to this chapter has been provided to clarify apparently contentious or unusual statements made in the data chapters. It has been written to be read as a single document if necessary, but it may also function as a hyperlinked reference for the reader primarily interested in the observational work and less interested in the theoretical framework.

I introduce and specify a number of concepts in this chapter which are not normally introduced into molecular or cell biological reasoning. I assure the reader both that these concepts are the minimal set I found necessary to pursue the scientific argument in the data, and that they are generated from good-faith interaction with the primary and secondary literature as an intimately interested participant (a colleague). In other words, I have made borrowings from philosophers, mathematically- and theoretically-inclined biologists, and so forth, in order to \textit{do my job properly}- I completed my formal biological education without ever having set foot in a philosophy or logic classroom, and I would have been \textit{entirely content} not to pursue remedial philosophical autodidaction. However, simply following the structure of recent papers published by leading labs rapidly led down the rabbit holes of model theory, causal metaphysics, and so on. Sorting out what the arguments of these papers are now \textit{requires} touching, at least briefly, on these sometimes obtuse subjects to understand the scientific logic of these modern "systems biology" studies. 

As I suggest below, how we understand the claims of "systems biology" has a large influence on how we structure our investigations into the phenomena themselves. If a researcher produces a sophisticated, well-fitted mathematical of some phenomenon, what sort of claims does the model support? How should we interact with the model if we have apparently counterfactual observational data? How do we identify whether the apparent conflict is a problem with the model or the underlying biological reasoning? How do we deal with interdisciplinary publications where the complete implications of the study's logic are not fully grasped by any one of the authors?

I have therefore sought to outline the methodology used in the data chapters as a type of guide to this study's reasoning for my own future reference, and for any colleagues coming out of various parts of the molecular biology tradition who are by now confronted with an astonishing variety of ways to interpret biological phenomena and few clear guidelines on how they might structure their research programs in light of the "systems" encounter. It is nevertheless by necessity confined to considerations relevant to retinal stem cells in zebrafish. I do not consider this an exemplary or "gold standard" methodology, nor a nominal catalogue of relevant, productive "systems considerations", even for this narrow topic. It is the attempt of one student to determine an appropriate method for one problem, in the context of vertiginous change within his field.

The final section of this chapter contains a number of suggestions with regard to the "futurology" implicit in some quarters of the molecular biological community's research program that are supported by evidence drawn from scientific fields that bear directly on the question of the constraints likely to be placed on planning. This was necessary because the biophysical constraints on biological practice are routinely ignored in methodological debates- we rarely make use of scientific ways of thinking about the likely context of our future scientific practice, and our decisionmaking suffers as a result. Where I have strayed outside of my own expertise, I have attempted to confine myself to citing well-supported, consensus views in those areas.


\subsection{Use of Hyperlinks within and between documents}

\subsection{Notes on style and conventions}

Unless otherwise noted, formatting of quoted material is preserved, so that italic emphases appear in the original.

\section{Considerations in Structuring the Investigation}

A scientific report is, by its nature, a retrospective structuring of an investigation. Structuring a scientific study requires defining the goals and methods of the study. To the extent that the products of a research program (eg. the intellectual output of a number of labs over some period of interaction) are structured by some common set of rules for determining their goals and methods, these rules turn out to be surprisingly difficult to define. As Philip Kitcher noted about the teaching of classical genetics:

\begin{longquote}

Neophytes are not taught (and never have been taught) a few fundamental theoretical laws from which genetic "theorems" are to be deduced. They are introduced to some technical terminology, which is used to advance a large amount of information about special organisms. Certain questions about heredity in these organisms are posed and answered. Those who understand the theory are those who know what questions are to be asked about hitherto unstudied examples, who know how to apply the technical language to the organisms involved in these examples, and
who can apply the patterns of reasoning which are to be instantiated in constructing answers. More simply, successful students grasp general patterns of reasoning which can be used to resolve new cases. \cite{Kitcher1984}

\end{longquote}

It is the effective use of these patterns of reasoning that constitutes the good or proper practice of biology. The learning biologist, if he inquires into the general structure of those patterns, is likely to be met with what Paul Feyerabend referred to as the "fairy-tale" underlying the special credibility afforded natural scientists:

\begin{longquote}

Scientists have ideas. And they have special methods for improving ideas. The theories of science have passed the test of method. They give a better account of the world than ideas which have not passed the test. \cite{Feyerabend1993}

\end{longquote}

This is an appealing myth, particularly for the learner motivated by a search for the truth, which may account for its use in redirecting the student to immersion in experimental practice. We often teac that there is actually only one method, "The Scientific Method" (hereafter Method), which consists in iteratively falsifying hypotheses about phenomenal reality, allowing a model of that reality (in the form of an logical argument constructed of these hypotheses) to be refined over time so that the model is, by correctly applying this procedure, inexorably brought closer to reality\footnote{This instruction, strangely, often fails to note the origin of this notion with Karl Popper and its subsequent shoddy reputation as a good description of, or prescription for, scientific activity.}.

The conscientious student, looking for serious academic treatments of the Method, is immediately forced to contend with a bewildering array of perspectives on what constitutes the Method and what procedures are legitimately employed in properly Scientific practice. Most of these are produced from accounts of the development of physics and its auxiliary sciences. Many insist that the Method consists of some formal criterion or procedure which is demonstrably not in use in biology or in scientific practice at large. The wide disagreement on the very concept of any such Method has, at least, the salutory function of disabusing the student of the notion that scientific practice could possibly be governed by the well-understood "special methods" of Feyerabend's "fairy-tale" scientists.

Deprived of any "special method" recipe to apply to his problem, the biologist only has resort to the patterns of reasoning in their field. These are normally understood heuristically or intuitively- we know how to make arguments about molecular systems without any training in formal logic, and without necessarily offering any account of \textit{what we are doing} when we are making the argument. This type of understanding is sufficient when one is simply applying these patterns to new cases. What are we to do when confronted with claims by a colleague who is using novel, foreign, imported, etc. scientific methodologies? We may attempt to "muddle through" by "seeing what works", but this concedes entirely too much to influences we generally understand to be orthogonal to systematisation of knowledge about nature. For instance, if "what works" is largely defined by what gets published, the result of "muddling through" will frequently be shaped more by the parochial interests involved in the scientific process than by the drive for better biology\footnote{For instance, both the academic careers of scientific personnel staffing with granting agencies and editorial boards and the commercial success of manufacturers selling expensive scientific equipment, reagents etc. depend directly on the perception that the methodology used by the personnel with the equipment and reagents is in some way \textit{orthodox}. This perception may be justified with reference to some philosophical rule to establish the boundary between science and non-science, such as Popper's "pseudoscience" demarcation. Thus a Health Canada toxicologist might be demoted, without much objection, for insisting that homeopathic cures be subjected to expensive clinical trials at great taxpayer expense (the toxicologist has incorrectly understood what standards are to be applied to orthodox and "pseudoscientific" heterodox traditions). To the scientific community's credit, unusual scientific views usually find airing \textit{somewhere}, but there are obvious "extra-scientific" sources of legitimacy that are used to promulgate particular views over others, even within what everyone recognises as scientific orthodoxy. For instance, the medical doctors interviewed for popular news and scientific media typically advocate for universal flu vaccination, e.g. "vax justification quote". The most authoritative meta-analyses of the biological evidence for flu vaccine evidence found "evidence of manipulation quote" and "low efficacy quote". In other words, flu vaccines appear not to be particularly effective treatments \textit{even in the context of pervasive manipulation of industry-generated evidence}. The conventional "scientific" wisdom that the population must be universally vaccinated against flu at great public expense therefore requires explanation. I suggest the only plausible explanations are inconsistent with the view of responsible scientific actors I offer below (eg. universal flu vaccination advocates have a good rationale for their position, but offering it to a distracted populace easily gulled by anti-vaccination conspiracy theory will not be as effective as simplistic suggestions of efficacy, so they concoct a white lie; universal flu vaccine advocates want to remain funded by vaccine manufacturers; medical practicioners are speaking without knowledge, etc.).}.

The student may therefore return to academic discussion of scientific practice looking not for any recipe-Method, or indeed systematic prescriptions about how to structure studies and analyse data, but simply for points of agreement on what science is and what scientists are doing. As a full survey of this literature would extend across a huge range of studies in a variety of disciplines, I have instead focused on one informative point of agreement arrived at in the philosophy of science. In the second half of the 20th century the observation was made by Thomas Kuhn, Paul Feyerabend, Imre Lakatos, and others, that science plainly did not proceed by iterative hypothesis falsficiation, as asserted by the more credible scientific realists of the first half of the 20th century. Kuhn, Feyerabend, and Lakatos conceived of different scientific "paradigms", "traditions", or "research programs", respectively, making competing claims on truth in a historical process of scientific development. Rather than collectively producing one huge cultural artefact called "Science", it became clear that the only way to explain the apparently "revolutionary" character of science, as well as the chaos governing succession of different scientific theories, was to understand science as the activity of people committed to a plurality of different methodologies interacting in particular historical contexts.

These ideas and their relation to one another took time to digest, but by the end of the 20th century, most philosophers of science agreed that scientific theories must be analysed in their historical, social, and intellectual context, and that diverse scientific schools of thought advance theories and models as competing explanations for phenomena, rather than proceeding by any mutually agreed upon Method. Scientists participate in and draw from these schools as they perform their work, convert from one school of thought on an issue to another as the latter becomes more fleshed-out and persuasive, and so on. From a biologist's perspective, this is a realistic description of scientific practice as a type of ecosystem, rather than the rote application of some philosopher's rule. The agreement of the philosophers suggests that this is genuinely a better description of what scientists are actually doing than the quasi-mythological "received view" of the past, particularly in the context of the bitter disputes on virtually every other topic.

I have therefore chosen to define a framework derived from these basic insights in order to concretely examine what I take to be the leading "research program" in my field.
 
\subsection{Study Structure in the Light of Tradition}

In order to structure our considerations of particular theories and evidence, I have chosen to appropriate a number of concepts from the philosophy of science: Feyerabend's "tradition", Lakatos' "resarch program", and Schaffner's "extended theory". I will briefly redefine these concepts for my own use before demonstrating how they inform the structure of this study.

\subsubsection{Feyerabend's Scientific Traditions}

The central, seminal insight of Paul Feyerabend's \textit{Against Method} is that the observed succession of scientific theories occurs by counterinductive advancement of an opposing theory, because only counterinductive comparisons \textit{between} theories are capable of showing up their implicit assumptions and allowing them to be challenged. Feyerabend explains:

\begin{longquote}

... it emerges that the evidence that might refute a
theory can often be unearthed only with the help of an incompatible
alternative: the advice (which goes back to Newton and which is still
very popular today) to use alternatives only when refutations have
already discredited the orthodox theory puts the cart before the
horse. Also, some of the most important formal properties of a theory
are found by contrast, and not by analysis. A scientist who wishes to
maximize the empirical content of the views he holds and who wants
to understand them as clearly as he possibly can must therefore
introduce other views; that is, he must adopt a \textit{pluralistic methodology}.
He must compare ideas with other ideas rather than with
'experience' and he must try to improve rather than discard the views
that have failed in the competition.\cite[p.20]{Feyerabend1993}

\end{longquote}

\textit{Against Method} takes as its historical exemplar Galileo's advancement of the heliocentric Copernican model of the sun/earth system against its orthodox Aristotlean competitor championed by the Catholic Church. Although we commonly think of the so-called Copernican Revolution as the succession of a properly formed theory from the empirical sciences overcoming an obviously defective theological explanation, Feyerabend shows that this conceals the actual means by which Galileo makes his persuasive case for the (itself badly defective) Copernican model. Centrally, it is process of comparing the heliocentric and geocentric theories that makes the implicit, unstated "natural assumptions" of the geocentric theory clear. Feyerabend elaborates on the kind of structures that counterinduction can reveal:

\begin{longquote}

Methodological rules speak of 'theories', 'observations' and 'experimental results' as if these were well-defined objects
whose properties are easy to evaluate and which are understood in
the same way by all scientists.

However, the material which a scientist \textit{actually} has at his disposal,
his laws, his experimental results, his mathematical techniques, his
epistemological prejudices, his attitude towards the absurd consequences of the theories which he accepts, is indeterminate in many
ways, ambiguous, \textit{and never fully separated from the historical background}. It is contaminated by principles which he does not know
and which, if known, would be extremely hard to test. Questionable
views on cognition, such as the view that our senses, used in normal
circumstances, give reliable information about the world, may invade
the observation language itself, constituting the observational terms
as well as the distinction between veridical and illusory appearance.
As a result, observation languages may become tied to older layers of
speculation which affect, in this roundabout fashion, even the most
progressive methodology. (Example: the absolute space-time frame
of classical physics which was codified and consecrated by Kant.)
The sensory impression, however simple, contains a component that
expresses the physiological reaction of the perceiving organism and
has no objective correlate. This 'subjective' component often merges
with the rest, and forms an unstructured whole which must be
subdivided from the outside with the help of counterinductive
procedures. (An example is the appearance of a fixed star to the
naked eye, which contains the effects of irradiation diffraction,
diffusion, restricted by the lateral inhibition of adjacent elements of
the retina and is further modified in the brain.) Finally, there are the
auxiliary premises which are needed for the derivation of testable
conclusions, and which occasionally form entire \textit{auxiliary sciences}.

...

Consideration of all these circumstances, of observation terms,
sensory core, auxiliary sciences, background speculation, suggest
that a theory may be inconsistent with the evidence, not because it is
incorrect, \textit{but because the evidence is contaminated}. The theory is
threatened because the evidence either contains unanalysed sensations which only partly correspond to external processes, or because
it is presented in terms of antiquated views, or because it is evaluated
with the help of backward auxiliary subjects.

...

It is this \textit{historico-physiological character of the evidence}, the fact that it
does not merely describe some objective state of affairs \textit{but also
expresses subjective, mythical, and long-forgotten views} concerning this
state of affairs, that forces us to take a fresh look at methodology. It
shows that it would be extremely imprudent to let the evidence judge
our theories directly and without any further ado. A straightforward
and unqualified judgement of theories by 'facts' is bound to eliminate
ideas \textit{simply because they do not fit into the framework of some older
cosmology}. Taking experimental results and observations for granted
and putting the burden of proof on the theory means taking the
observational ideology for granted without having ever examined it.

\cite[p.52]{Feyerabend1993}
\end{longquote}

Feyerabend argues that it was the counterposition of the Copernican model with orthodox geocentric and geostatic Aristotlean conceptions that showed up precisely this kind of implicit assumption which gave rise to the unchallengeable "fact" that the Earth could not be moving rapidly through space because objects are observed to fall straight down- if the Earth were in motion, falling objects would appear to be moving in a slanting trajectory.

\begin{longquote}
We start with two conceptual sub-systems of 'ordinary' thought ... One of them regards motion as an absolute process which always has effects, effects on our senses included.
The description of this conceptual system given here may be somewhat
idealized; but the arguments of Copernicus' opponents, which are
quoted by Galileo himself and, according to him, are 'very
plausible', show that there was a widespread tendency to think in its
terms, and that this tendency was a serious obstacle to the discussion
of alternative ideas.

...

The second conceptual system [the Copernican model] is built around the relativity of
motion, and is also well-entrenched in its own domain of application.
Galileo aims at replacing the first system by the second in all cases,
terrestrial as well as celestial. Naive realism with respect to motion is
to be completely eliminated.

...

Viewing natural phenomena in this way leads to a re-evaluation of all
experience, as we have seen. We can now add that it leads to the
invention of a \textit{new kind of experience} that is not only more sophisticated
\textit{but also far more speculative than} the experience of Aristotle or of
common sense. Speaking paradoxically, but not incorrectly, one may
say that \textit{Galileo invents an experience that has metaphysical ingredients}. It
is by means of such an experience that the transition from a geostatic
cosmology to the point of view of Copernicus and Kepler is
achieved.

\cite[p.69]{Feyerabend1993}
\end{longquote}

In other words, the conceptual frame of the Galilean observer has shifted so that fundamental, common-sense natural perceptions (objects which are not observed to be in motion cannot be in motion) are overturned, and "experience now ceases to be the unchangeable fundament which it is both in common sense and in the Aristotelian philosophy.
The attempt to support Copernicus makes experience 'fluid' in the
very same manner in which it makes the heavens fluid, 'so that each
star roves around in it by itself. An empiricist who starts from
experience, and builds on it without ever looking back, now loses the
very ground on which he stands." \cite[p.72]{Feyerabend1993}

Feyerabend thus shows that the success of the Copernican Revolution was a result of counterinductive comparison of the older Aristotlean explanation with Galileo's theory involving relative motion that revealed the flawed natural assumption of the geostatic model- the assumption that all motion is operative, that objects observed not moving with respect to an observer are indeed at rest. Only by thinking of motion as a relative phenomenon that only produces observable effects when bodies are moving \textit{with respect to one another}, does the problem with the assumption of absolute motion become obvious.

The success of the Copernican tradition over the Aristotlean one was therefore not the result of any of the usual "rules" assigned to the Method. Galileo's theory substantially contradicted the available evidence, erroneously asserted the reliability of telescopic observations, made extensive use of ad hoc hypotheses, and was advanced by propagandistic and even dishonest means. As Feyerabend notes, if any of the typical suggested Method criteria were applied, the Church would have won the debate and we would still have an Aristotlean cosmology!



\subsubsection{Lakatos and Schaffner on the Resarch Program and Extended Theories}









\subsection{Responsible Actors}

\section{Determinism, Stochasticity, Free Will and Choice}
\subsection{Free Will and Choice}
\label{sec:choice}
