
\section{The Cartesian Fracture: A Possible History of the Systems Biology Encounter}

Having a clearer idea of how we might analyse the "local" theoretical structures produced by the SCBT, we must now give some consideration to the metaphysical terrain in which the practices of the SCBT are situated. This requires making short excursions, that should nonetheless be familiar vistas for older biologists, if not often viewed in this light. The kind of eye we cast upon this landscape as we traverse it is given to us by Feyerabend- we must be sensitive to what Feyerabend called "natural assumptions". As I note in \hyperref[Ch2]{Chapter 2}, we have a good candidate which appears in the language of Harris' papers- Agency.

Throughout Harris' work, we hear of retinal progenitors that make "choices". When we look carefully at the "choice", the interpreter and its relationship to the "choices" afforded by the cell's internal or external dynamics, it seems unlikely that we can understand the phenomena Harris is describing without some sense of what is "choosing," and what, in molecular terms, this means. That is to say, whether or not "choice" is merely a metaphor, reducible to the random jitterings of proteins in a membranous sac, whether or not this reduction is possible or consistent, we still require \textit{some} account of the macromolecular constituents that constitute a cell "choosing," or it should not be spoken of at all. I would say that a model of the chooser (the progenitor) choosing (undergoing whatever internal dynamics result in the specific lineage that it produces) is in fact \textit{what we are after}, and only secondarily, the specification of potential inputs into this choice.

Now, it is at least a little surprising that what is, ostensibly, a metaphor, is in fact a sort of unspecified void in the EHJMEx laid out to describe Harris' work- it requires substantial interpretation to find plausible candidates. This sort of explanatory void calls for explanation, particularly when it crops up again and again in different biological subtraditions. At this point, we should avoid falling into simplistic arguments about reductionism to take account of what our "ancestors" in the MBT were trying to do, and in what context. Let us take traditionally-educated MBT practitioner-\textit{cum}-philosopher Michel Morange as our guide here:

\begin{longquote}
Different periodizations of molecular biology have been proposed by
historians. Some consider that what characterizes molecular biology and
distinguishes it from disciplines that predated it, such as biochemistry
and genetics, is the informational vision with which it is pervaded. Such
a model is supported by the abundance of informational expressions
within molecular biology. In this historical account, molecular biology
was a consequence of the Second World War, of the information theory
and computer science which were developed to answer the needs of
communication, and of computing generated by the war. I will adopt
another periodization supported by Lily Kay in her pioneering book (Kay
1993) and by many other historians, positioning the birth of molecular
biology earlier, in the 1930s. The rise of molecular biology found its
initial impetus in the development of new technologies that allowed the
study of a domain of the living world which had previously been partially
absent from the descriptions. It stretched between the molecules studied
by organic chemists and the cellular structures barely visible under the
light microscope. This was the domain of macromolecules, so named
when the colloid theory progressively vanished in the 1920s and the
existence of macromolecules was clearly demonstrated.

But these technological developments were concomitant with an effort
to "naturalize" the phenomena of life, to uncover the veil of mystery
that had so far covered the intimate functioning of organisms. The sheer
ignorance of what constituted "living" was considered as an intellectual
scandal by many scientists, in particular physicists. This scandal became
more obvious with the rapid progress made in physics in the three first
decades of the twentieth century: matter and energy had been fully
naturalized. The same had to occur for life.

Can these ambitions be in any way called "reductionist?" The fact
that physics was considered as a model and the proclaimed intent
was to naturalize organismic phenomena, are not sufficient reasons to
answer in the affirmative. Even the development of technologies aiming
at describing what happened at the level of macromolecules cannot
be considered per se as a reductionist program. The objective was to
obtain information on a domain which had been so far a "black box,"
and obscured by the models of colloidal chemistry. It did not mean that
this level would be the "explanatory" level. Its increasing place was
not preformed in the early efforts to describe it. In fact, there was no
agreement among the founders of molecular biology on the level and
way in which living phenomena had to be naturalized. The reductionist
approach of crystallographers, such as Max Perutz, was in sharp contrast
to the ambitions of Max Delbr{\"u}ck, one of the founders of the phage
group, who had set out to discover new laws that would explain global
phenomena specific to life, such as reproduction (Fischer and Lipson
1988).

To acknowledge better this diversity of attitudes among the first
molecular biologists, I examine the case of two French biologists who
played a major role in the development of molecular biology and whose
accomplishments reached international visibility, Boris Ephrussi and
Jacques Monod. Ephrussi made decisive contributions, from his early
work with George Beadle, which finally led to the "one gene - one
enzyme" relationship (later established by Beadle and Edward Tatum),
to the establishment of mitochondrial genetics, the development of cell
fusion technique, and the early study of embryonic stem cells (derived
from teratocarcinoma; Morange 2008a). He was an admirer of the
epistemological principles of physics, of the ability to derive laws from
simple general principles, and afforded to genetics a preeminent place in
the development of biology during the twentieth century precisely for the
reason that this discipline shared with physics the same characteristic of
abstraction (Ephrussi 1979). Nevertheless, throughout his life Ephrussi
remained an embryologist and at the end of his career he vehemently
opposed the direct application of the genetic regulatory models derived
from the study of microorganisms to the explanation of differentiation
and development. There is no sign of a simplistic reductionist view in
Ephrussi's work.

The same is true for Monod, co-author with François Jacob of the
operon model. Monod was also convinced of the superior epistemological
value of physics, and believed that principles preeminent in physical
explanations, such as symmetry, had their place in biology. This
conviction led him to propose the sophisticated allosteric model to
account for the characteristics of regulatory enzymes. These proteins
were formed of multiple identical subunits that might exist under two
different conformations. The transition between these two different
conformations was coordinated in order to maintain the symmetry of
the macromolecule. Monod also considered that the place occupied by
teleological arguments in biology - the justification for the existence of
structures by the "functions" they fulfill in organisms - was a scandal,
and for him the true motivation to turn toward biology was to overcome
 it (Judson 1979). Reductionist approaches were simply one way to lift
the veil of mystery surrounding organisms, a methodology that had been
eminently successful in physical sciences.

Therefore, molecular biology was not the result of a reductionist
program, but an attempt of \textit{aggiornamento}\footnote{Morange here uses the Italian term, meaning modernisation or updating, and usually associated with the Second Vatican Council, interestingly. I am unsure whether Morange intends to invoke the mixed success of Vatican II, but the reference to the updating of an orthodox traditional canon is apt.} of biology, by reducing the
gap existing in its descriptions (between simple molecules and subcellular
structures), and by employing the epistemology of physics as a model.
Moreover, historical models that give one institution or one institute a
major role in this story (the Rockefeller Foundation and Caltech for Kay)
are obviously much too narrow in their perspectives (Kay 1993). Other
institutions created similar research programs in countries other than
the United States, all with the same objectives, and quite independently
from the Rockefeller Foundation. One outstanding case was the Institute
of Physico-Chemical Biology (IBPC) created in France by Jean Perrin
and Baron Edmond de Rothschild, with the financial support of the
latter's foundation, to bring together physicists, chemists, and biologists
in efforts aimed at unravelling the fundamental mechanisms operating
in organisms. The development of new technologies derived from
physical knowledge constituted a part of the research program at the
IBPC which, from its creation right up to the Second World War, was
an active player in what in retrospect can be called "molecular biology"
(Morange 2002).
\cite{Morange2008}
\end{longquote}
 
In other words, the MBT was, from its inception, intended to bring the rigour of physics into biology. Morange's point, that this was not some simpleminded program of reducing life to its physico-chemical properties, but rather one with a diverse range of opinion on what MBT explanations should look like, is well taken. However, let us consider Monod's words themselves:

\begin{longquote}
 Biologists of the early part of this century-some biologists, the school of Jacques Loeb, for instance-
certainly had this ambition, quite decidedly. They were philosophical reduc-
tionists and were convinced that the living could be reduced in principle to
physics and chemistry, and nothing more.
 "So that's not new," Monod said. "It's of course present in
molecular biology, but there's nothing new in it."

What is new in molecular biology is the recognition that the essential
properties of living beings could be interpreted in terms of the structures of
their macromolecules. This, you see, is much more specific-and in fact it part­
ly contradicts the hope of the physical-chemical school of the beginning of the
century. The biologists of that time-and the time extended into the days
when I was a student, so I know this quite well-believed, to put it roughly,
that the laws of gases would explain the living beings. That is to say, metabo­
lism in the cells would be explained by the general laws of chemistry. This
was natural because it was the time when great advances were being made in
the understanding of the behaviour of substances in solutions, and of semi­
permeable membranes, and so on."

...

"But the general point is that if the cell, for regulation, really did have at its disposal only direct chemical interactions, understandable by the general laws
of chemistry, then the overall tendency of the system would be to go to equilibrium. Now chemical equilibrium means death," Monod said. "The cell survives only by being very far off equilibrium. What we now understand is
quite the reverse: the cell is entirely a cybernetic feedback system. The regulation is entirely due to a certain kind of circuitry like an advanced electronic
circuit. But it is a chemical circuitry-and yet it transcends chemistry. It is indirect. It enables the cell to gain a degree of liberty from the extreme stringencies of direct, general chemical interactions. And it works virtually without
any expenditure of energy. For example, a relay system that operates a mod-
em industrial chemical factory is something that consumes almost no energy
at all as compared to the flux of energy that goes through the main chemical
transformations that the factory carries out. You have an exact logical equiv-
alence between these two-the factory and the cell. This effect is entirely
written in the structure of the proteins, which is itself written in the DNA.
And therefore is available to be selected for, and to evolve.
\cite{Judson1996}
\end{longquote}

Let us take a moment to marvel at this; Monod's understanding of the way forward in molecular biology, given in an interview in the 1970s, is \textit{almost precisely the same} as Keller's cited above, given in 2000, and Fagan's, given in 2015. It is essentially identical to the views held by the eminent Systems theorists Stuart Kauffman and Sui Huang as late as 20XX. There is something very wrong here. Let us be frank: Monod's knowledge of the molecular basis of life was utterly rudimentary at the time that he had formulated his cybernetic view of cellular regulation. How is it possible that \textit{nothing has changed} in almost half a century? How is it possible that Monod's simpleminded abstract diagrams of genetic regulation are \textit{still} being peddled as the "Systems" resolution to our explanatory problems? If this approach was truly the way forward, how is it possible that it has borne no fruit after the sincere application of so much intellectual firepower? Let's return to Morange to take in the orthodox view:

\begin{longquote}
By contrast, it is probably more appropriate to consider the rise of systems and
synthetic biology as the last step in the
project of early molecular biologists to
‘naturalize’ the organic world—that is, to
provide natural explanations of biological
phenomena and to weed out teleological explanations, the mere existence of
which was considered to be a scandal by
prominent molecular biologists such as
the French biologist and nobel laureate
Jacques Monod (1910-1976; Morange,
2008). the development of synthetic biology, including some of its most ambitious
projects, can be considered as the last step
in this naturalization process. the best way
to demonstrate that the ‘mystery’ has been
definitively banished from the realm of
organisms would be to synthesize a living
organism ‘from scratch’—from inorganic
and organic components.
\cite{Morange2009}
\end{longquote}

That is, the rise of the "Systems" disciplines represents, in some way, the apotheosis of Monod's programme. If it succeeds, molecular biology will be free of "weedy" teleological explanations\footnote{One cannot help but feel that the weed metaphor here expresses both the unwantedness of these explanations as well as their stubborn persistence and robust proliferation.}, mystery will be banished, and we will gain the same kind of full synthetic control over life that we now take for granted in our nanoscale engineering practices.

On this view, Monod was an incredible visionary: nothing needs to be altered about his basic theoretical approach to explaining biological phenomena. We simply need not to repeat the mistakes of Keller's gene-fixated reductionists, and to apply Monod's cybernetic method faithfully, and we will arrive at a full understanding of biological systems. The appearance of the language of choice and agency in Harris' work is merely an inappropriate metaphor, the inability to identify molecular dynamics that might plausibly constitute a chooser, just a sign of the immaturity of our auxiliary technical sciences.

Being a mere benchworker, and not a theorist, I have a certain bias toward thinking that perhaps the problem is \textit{not} a problem with the technical means or those employing them, but rather with those who have become so removed from our actual practice that they cannot see where those methods are taking us. In other words, perhaps Harris knows something that Monod could not have realised: the \textit{aggiornamento} is a failure. End directedness, telos, are \textit{fundamental features of life}. They cannot be elided, and they will not be adequately explained by any variety of physical explanation, nor by cybernetic wiring diagrams. Monod's program was always doomed to fail, and it was because he was unable to interrogate his own metaphysical assumptions that he was unable to see this.

\subsection{Descartes' Cogito and the Metaphysical Fracture}



While Morange's location of the MBT's origins in the 1930s seems like a good one, as Feyerabend teaches us, the most important traditional assumptions are precisely those that have been long-forgotten. It is these roots of our tradition that must be excavated if we are to see why the tree has grown crooked.








\subsubsection{Metaphysical Crisis and Failure of Neo-Darwinian Synthetic Orthodoxy}

Briefly, the Neo-Darwinian Synthetic Orthodoxy (NDSO) consists of the view that biological phenomena are explicable solely in terms of natural selection operating on molecular genetic material, and that all explanations of biological systems are reducible to these terms. This is a common default view in the MBT, though properly speaking it is a type of hybrid MBT/CGT theory. Macromolecular MEx are therefore taken to provide complete, adequate explanations for phenomena at all levels of biological organisation. There is therefore little need for a neuroscientist to look anywhere other than macromolecular MEx and evolutionary population biology to explain, for instance, organised eukaryotic behaviour, and even less the behaviour of prokaryotes or individual eukaryotic cells. To the extent that phenomena like culture are considered at all, they are done so under some direct (but illicit) analogy to the population behaviour of genes under natural selection, ie. memes.

Now, it has long been recognised that the NDSO has an extremely serious logical problem, which is its circular explanation for organised, end-directed behaviour. This arises as a consequence of its Darwinian traditional heritage, for it is committed to the situational logic which entails selection. That is, a particular situation must have certain conditions that obtain before Darwinian selection can operate. These conditions are as follows:

\begin{enumerate}
\item Living things must vary and this variation must have a heritable basis.
\item Resources available to living things must be limited.
\item Living things must strive to maximise consumption of these limited resources (there must be, in other words, Malthusian population pressure).
\cite{Darwin1888, Swenson1999}
\end{enumerate}

For MBT practitioners, the first two conditions are unremarkable entailments of reproduction and the finite nature of material things. The third is rarely carefully considered in most biomedical contexts outside of bacterial work. It is, however, a critical point, and Darwin makes it explicitly throughout "The origin of species", as when he states:

\begin{longquote}
In looking at Nature, it is most necessary to keep the foregoing
considerations always in mind—never to forget that every single
organic being may be said to be striving to the utmost to increase
in numbers; that each lives by a struggle at some period of its life;
that heavy destruction inevitably falls either on the young or old,
during each generation or at recurrent intervals. Lighten any
check, mitigate the destruction ever so little, and the number of
the species will almost instantaneously increase to any amount.
\cite[p.52]{Darwin1888}
\end{longquote}

In other words, striving toward Malthusian geometric increase is the default state of living things, which would be realised in actual population numbers were resources not limited. The obvious objection that life does not uniformly consist of single-minded population-maximising proliferation is dealt with by invoking further intentional dynamics:

\begin{longquote}
But the real importance of a large number of eggs or
seeds is to make up for much destruction at some period of life;
and this period in the great majority of cases is an early one. If
an animal can in any way protect its own eggs or young, a small
number may be produced, and yet the average stock be fully kept
up; but if many eggs or young are destroyed, many must be
produced, or the species will become extinct.
\cite[p.52]{Darwin1888}
\end{longquote}

That is, if an organism has access to the behavioural and organisational sophistication required to create a local niche of less vicious selective pressure, this is the intentional dynamic which explains its failure to "strive to the utmost to increase in numbers" in an absolute sense; it has traded off fecundity for survival, its "utmost striving" consists partially in its expenditure of finite time, energy, and nutrients arranging the required protection for offspring, not only efforts to achieve brute numerical increase.

Therefore, a particular logic (selection) is understood to necessarily follow from conditions which include the "utmost striving" of organisms, their protective behaviour towards their offspring, and so on. This "striving" is nothing other than the end-directedness of living things, the tendency of any bearer of Life to organise their behaviour in a way that allows them to feed, metabolise, reproduce, and so on. That is to say, agency is \textit{presupposed} by the explanation for why selection occurs in the first place.

If we still doubt that agency is what is implied by this Malthusian principle, we should take heed of the conclusion of Holt's 2009 thought experiment demonstrating that there are fundamental spatial limitations to geometric growth:

\begin{longquote}
In discussions of exponential growth, everyone recognizes the assumption that resources are not limiting, but what is usually left unstated is that because of conservation
of mass and energy, movement at the level of individuals, or their offspring, and their
material constituents, is absolutely required for organisms and their descendents to not
be resource-limited over even a single generation. A hidden assumption in the exponential growth “law” is thus that organisms must become spatially dispersed as they grow,
reproduce, and die, relative to the resource base that they require if they are to construct
and maintain their bodies as well as reproduce daughter bodies. Considering processes
in space is not a complicating “add-on” to evolutionary theory, because demography
must be grounded in space, and movement is a ceaseless and inescapable dimension of
the ferment of life.
\cite{Holt2009}
\end{longquote}

In other words, it is impossible for the condition of geometric growth to obtain in ecosystems unless organisms are free to disperse, and the directed strategies that we find to do this (whether the production of seeds with wind-catching plumes or deliberate animal motility organised around sensory patterns) are therefore presupposed by and beyond the explanatory reach of Darwinian theory.

We may briefly wonder whether the theories of our contemporary Darwinian population geneticists (whom we so rarely take time to question on their own traditional metaphysics) really depend as heavily on Malthus as did Darwin. As Lockwood notes in his vigorous criticism of the work of ecological theorists fixated on producing "laws of ecology":

\begin{longquote}
One of the most cited candidate ecological laws is Malthusian Growth (Ginzburg
1986; Brown 1997; Turchin 2001; Berryman 2003). Simply put, the law states that
a population will grow or decline exponentially (or geometrically), provided the en-
vironment remains constant for all individuals and that sufficient resources exist for
the population. Mathematically, this verbal description has a simple formulation,

\begin{equation}
N(t) = N_0e^{rt}
\end{equation}

The number of individuals, $N$, at time, $t$, is a function of the number of individuals at time 0, $N_0$, and the growth rate, $r$.
\cite{Lockwood2008}
\end{longquote}

Thus we find that this commitment remains so prevalent that there remains an active contingent who wish to enshrine it as a biological equivalent of one of Newton's laws! Lockwood points out that no empirically-informed model suggests geometric growth of single, isolated populations \cite{Lockwood2008}. If such a principle is operable, it is perhaps only expressed in the case of some ideal limit \cite{Holt2009}, one where spatial considerations are less relevant than usual, with nutrients and energy available in massive excess- a newly inoculated bacterial flask in a shaker, or a human population systematically exploiting fossil fuels for the first time, perhaps.

Obviously, then, intentional behaviour is inexplicable in NDSO terms. Agency cannot be simultaneously required for selection to occur in the first place, and also be explicable solely in terms of the operations of selection. The explanation is circular and results in a historically impossible bootstrapping situation where the development of life on Earth could never have begun.

We should pause for a moment to reflect on the dependence of all Darwinian evolutionary theory on a single postulate Thomas Malthus introduced in order to argue that there were biophysical constraints to the \textit{political} pursuit of arbitrary social perfection:

\begin{longquote}
This natural inequality of the two powers of population and of
production in the earth, and that great law of our nature which must
constantly keep their effects equal, form the great difficulty that to me
appears insurmountable in the way to the perfectibility of society. All
other arguments are of slight and subordinate consideration in
comparison of this. I see no way by which man can escape from the
weight of this law which pervades all animated nature. No fancied
equality, no agrarian regulations in their utmost extent, could remove
the pressure of it even for a single century. And it appears, therefore, to
be decisive against the possible existence of a society, all the members
of which should live in ease, happiness, and comparative leisure; and
feel no anxiety about providing the means of subsistence for
themselves and families.
\cite{Malthus}[p.5]
\end{longquote}

In effect suggesting that the Star Trek utopias of his day were likely to run face first into the finitude of natural resources, Malthus was using a rhetorical tactic to advance an \textit{explicitly} normative and political argument (he had no data to suggest that geometric population growth is typical- this is not an scientific explanation!). Darwin was working in a scientific environment enamoured of maximalisation principles, and was notably influenced by British political economy and scientific agronomy, both of which emphasized maximalisation of productive factors. In particular, Adam Smith's principle that self-interested actors pursuing their own interests in a fully liberalised \textit{laissez-faire} economy maximised overall returns did much to popularise this approach.

If we take the NDSO to be a mixed body of EBT/MBT theory produced by the sustained interaction between these traditions, we should be aware of the possibility that the logical defect we have identified here is a consequence of usually-unrecognised (for MBT practitioners) traditional commitments to EBT metaphysical principles (even, perhaps, "laws") that are unsupported by empirical evidence, and may indeed merely be part of the \textit{historico-physiological character} of Darwin's theory, influenced as he was by his scholarly environment.

\subsubsection{Importance of NDSO Metaphysical Failure and the Need For Monist Scientific Ontology}

While the circularity of NDSO explanations for biological agency is concerning in and of itself, it is one of a broader class of metaphysical errors arising from Cartesian dualism. In a general sense, Descartes' bifurcation of reality between (active, living, end-directed, meaningful) "Mind" and (passive, dead, meaningless) "Matter" resulted in the problems which form the background to the \hyperref[SBE]{SBE}. Robert Swenson describes this well:

\begin{longquote}
Historically, the problem in modern science is traced to Descartes's
dualist metaphysics where physics and psychology were literally defined
by their mutual exclusivity. The physical part of the world was taken to be
exhaustively defined by extension in space and time, and consist of
reversible, qualityless, inert or 'dead' particles ('matter') governed by
efficient cause and rigid deterministic law from which the active, striving
immaterial 'mind' (the self, 'thinking Γ, or psychological part) without
spatial or temporal dimension, was said to be immune. The physical
world, thus described, incapable of ordering itself, and extensionally
defined, had to have order, intentionality, and meaning imposed on it
from outside by 'mind'. Cartesian incommensurability came full-blown
into biology through Kant and with the later rise of Darwinian
evolutionary theory which was based on the assumptions of
Boltzmann's thermodynamics. Recognizing that the active, end-directed
striving of living things could not be accounted for from within the 'dead'
world of physics Kant, arguing for the autonomy of biology from physics
rather than recognizing a need for a revision of physics, promoted a
second major dualism, the dualism between biology and physics, or
between living things in general and their environments. Boltzmann's view
of the second law of thermodynamics as a universal law of disorder
provided the basis for fully entrenching the postulates of incommensurability in supposed physical law. The transition from disorder to order,
he asserted, was infinitely improbable. The world came to be seen as two
incommensurable 'rivers', the river of physics flowing down to disorder,
and the river of biology and psychology, the 'river' comprising the
epistemic dimension of the world or the intentional dynamics of living
things, flowing up to increasingly higher levels of order. The active,
directedness towards (or 'aboutness'), or telos constituting the intentionality of the epistemic act, and the semantic content on which it depends,
were seen not only as outside of physical law, but entirely anomalous with
respect to it.
\cite{Swenson1999}
\end{longquote}

It is precisely the lack of a good explanation for the "upward flowing river", which produces biological order and diversity out of entropic chaos, that is at the core of the problems within the MBT that gave rise to the impetus to pursue "Systems" approaches. The very same biological end-directedness that Darwin had to assume axiomatically, in order to make the logic of selection work, bedevilled pre-molecular developmental biologists:

\begin{longquote}
Unfortunately, notwithstanding the impressive advances
of nineteenth-century biological science in unraveling the
physicochemical basis of such physiological processes as respiration and metabolism, little if any progress could be reported on the problem of organization. And toward the
close of that century, even Claude Bernard-the man who
had contributed so much to our understanding of the
chemistry of physiology-seemed ready to despair. In 1878
Bernard wrote, "There is a kind of pre-established design for
each being and each organ, so that, considered in isolation,
each phenomenon of the harmonious arrangement depends
on the general forces of nature, but taken in relationship
with the others, it reveals a special bond: some invisible
guide seems to direct [the phenomenon] along the path it
follows, leading it to the place it occupies."
\cite[p.108]{Keller2000}
\end{longquote}

The solution to this problem in the MBT was to locate the agency of the cells "self-organising" to form a tissue in macromolecules themselves:

\begin{longquote}
But the twentieth century proved notably more beneficent. With the advent of molecular biology, biologists seemed finally to have found their homunculus, and it
turned out to be, after all, a molecule. Rereading Bernard’s
words of 1878 with the benefit of less than one hundred
years of hindsight, François Jacob could write, 
There is not a word that needs be changed in these lines today: they contain nothing which modern biology cannot endorse.
 Jacob’s solution of this age-old impasse was ... the genetic program. Here, buried deep inside the innermost core of cellular structure, inscribed "in a sequence of chemical radicals," was the "invisible guide" required to direct the organism "along the path it follows,
leading it to the place it occupies."
\cite[p.109]{Keller2000}
\end{longquote}

As Keller relates in her classic history of 20th century genetics, the phenomenon of genetic redundancy that came to the fore in the 1990s seriously challenged


I regard this as one of, if not \textit{the} most important cognitive defect among MBT practitioners. Aside from the logical error of circularity, the NDSO is part and parcel of broken Cartesian dualist ontology, which divides reality into ideal Mind and lifeless Matter, unable to explain how one could ever affect the other\footnote{The consequences of this are dire; an entire generation of young biologists are partially under the sway of ultra-Darwinist NDSO extremists like Richard Dawkins and Daniel Dennett, who literally deny that their own work is anything other than the product of ideal "replicators" (genes, memes) hosted by dead matter, and intended only to promote the replication of their own genomes and memetic concepts. Why someone who openly confesses that their statements have no meaning beyond increasing their own sexual success, or the sales of their books, should be taken seriously is beyond me.}. While Cartesian dualism was the philosophy that underpinned the technical scaffolding of the Industrial Revolution, and hence made the MBT possible, it has been the source of enormous confusion and is the fundamental roadblock to achieving better understanding of how complex biological systems like tissues can arise from the coordinated, end-directed behaviour of their constituent agents (cells).

\subsubsection{Pathologies of Cartesian Dualism}

\begin{longquote}
It is the epistemic relations between living things and their
environments, or the 'intentional dynamics' of living things, to use Shaw's
felicitous term (e.g., Shaw et al. 1992), defined here as 'end-directed
behavior prospectively controlled or determined by meaning or "informa-
tion about" ' (Swenson 1997a: 3), that effectively distinguishes the dynam-
ics of the living in the most general sense from non-living. The emphasis
is on relations between living things and their environments because it is
through these relations that intentionality comes to be constituted and
in the context of which semantic content is found. It is the Organism-
environment interface', as Hoffmeyer (in press) has put it, that must 'be
placed at the center of evolutionary theory' if the epistemic dimension is to
be understood, and it is just this interface, 'the interface between physics,
psychology, and biology' (Swenson 1997a: 3), that the Cartesian postu-
lates of incommensurability (the first between psychology and physics and
the second between biology and physics) preclude from being understood.
A principled account of intentionality, a theory that can show what it is,
in effect, to the world, and provide the basis for semantic content or
meaning, must be an account that can dissolve these postulates.
\cite{Swenson1999}
\end{longquote}


\section{Mathematical concepts encountered in MEx}
\subsection{Dynamical Systems Theory}
\subsection{Probability Distribution Function}
\label{PDF}
\subsection{Ordinary Differential Equations and Dynamical Systems Theory}
\label{SODEs}

\subsection{Expression of Traditional SCBT Models As Agent Models}
\subsection{From Autoreferential Asemiotic Agents (AAA) to Semiotic Agents (SA)}
\subsection{Cellular Semiosis: Phenomenal Logic and Relation to HJMEx}