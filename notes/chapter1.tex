\chapter{Contextual Introduction and Theoretical Background}
\section{"Systems Biology" in 2018}
\subsection{Sasai's Result and a Student's Perplexity}

In 2012, the late Yoshiki Sasai and his research group published a remarkable report, describing the production of an optic cup (that is, a developmentally primordial eye) from isolated embryonic stem cells in culture, treated with a limited array of exogenous factors \cite{Nakano2012}. Not long after enthusiastically reading this paper, I encountered a senior faculty member who, likewise excited about the result, asked me what I had thought about it. Without considering my reply, I blurted that I was amazed that the production of the complex three-dimensional structure of a retina could somehow emerge out of a "program" intrinsic to the stem cells themselves, in the absence of cues or support from any surrounding tissues. What I had taken to be a casual conversation suddenly became the kind of heated interrogation undergone by students when their teachers come to suspect that their charges have not, in fact, absorbed any of their instruction after all. I was flabbergasted when the instructor, whose views on developmental biology I thought I had thoroughly absorbed, insisted that the central message of Sasai's result was that the retinal structure was \textit{not} in any sense "programmed" into the stem cells. Conceding that the "program" metaphor was inappropriate, I nevertheless insisted that the structure had to emerge, at least in part, out of some property borne by the stem cells themselves- how else could an isolated, disordered clump of undifferentiated cells produce a complex laminar retina? Perhaps sensing my increasing perplexity, my interlocutor relented and left me to stew in a rapidly deepening morass of seemingly intractable questions.

I could accept the obvious truth that the three-dimensional structure of a retina was in no sense "coded for" by any macromolecular constituent of the cells. Sasai's group seemed to have demonstrated, however, that it was not possible to ascribe the production of this intricate biological structure to ordered interactions between the cells and their environment- the cells were free of anything resembling an organismal developmental context. This was indeed, as the paper's title suggested, a case of "self-formation". It was easy enough simply to insist that the system "self-organised", that structural order "emerged" from lower hierarchical levels of biological activity that did not themselves contain any explicit structural information. That said, there seemed to be no way to explain this "emergence" in terms of the molecular constituents I had been taught to think of as fundamental to all biological systems, or even to the cellular behaviours which I could, at least, clearly understand as "emerging" from the activity of those macromolecules. Moreover, how would this emergent property be subject to selection, to produce the functional properties of the retina? Given the growing body of evidence suggesting that the behaviour of stem and progenitor cells is, in some sense, fundamentally stochastic \cite{Chang2008}, how could the reliable "bootstrapping" of stem cells into the same ordered structure, in culture after culture, be explained? Even if an enumeration of the required macromolecular "machinery" or cellular behaviours could be supplied, would this actually constitute an \textit{explanatory} account of the emergent phenomenon? What was the sense in which this "self-formation" was a scientific concept subject to empirical scrutiny? Beyond this, was the production of such an explanatory account actually a valuable objective worth pursuing? Sasai had, if anything, just clearly demonstrated that one need not understand anything about the underlying processes governing the formation of retinal structures in order to put them to work.

These questions were, for me, not abstract- they bore directly on my research into the activity of retinal stem cells in zebrafish. However, after some years, I began to understand that the themes expressed by these questions, and indeed, the mutual incomprehension of the conversation which gave rise to them, were not merely features of stem cell biology, but rather products of the present state of disunity in biological inquiry as a whole. It was clear to everyone that the intellectual background that we all shared, orthodox 20th century molecular biology, was in some way unable to effectively address the complexity being documented by high-throughput -omics experiments and wide-ranging comparative studies. A symptom of the institutional response to this challenge is visible on the cover of this thesis- this work is submitted under the departmental umbrella of "Cell and Systems Biology". "Systems biology" is now a pervasive formulation, the term "Systems" being derived from complexity theory, which studies complex systems, so-called because of the difficulty in modelling them. Over the last two decades there has been a profusion of new departments, journals, endowed chairs, and so on with this title. "Molecular biology" is, by contrast, now vanishingly rare as a descriptor \cite{Morange2008}. However, much of what blandly passes under the monolithic banner of "systems biology" is, in fact, the product of several competing schools of thought, with entirely different and mutually exclusive conceptual frameworks for interrogating something like a retina. It therefore became necessary to place my own work within this much broader theoretical context in order to determine how best to interpret and evaluate it. I have therefore organised this thesis into three chapters, the first of which briefly deals with this broad theoretical background, providing the context for the approach taken in the second and third chapters, treating the specific research problems.

\subsection{Progress, Crisis, and the Rise of "Systems"}

The cognitive appeal of molecular biology is enormous. The explanatory power of this research program, which we may define as the effort to explain biological systems in terms of macromolecules (at a variety of levels of organisation), remains unrivalled in the biological sciences. Virtually all of the concrete, material answers that the biological sciences now provide regarding what life is and how it functions are derived from or fleshed out by this body of work. The mechanistic vision of biological function, in which cellular macromolecules are understood to be nanomachines with functionally defined parts (e.g. peptide motifs) making up "factories" in precisely bounded subcellular compartments, is pervasive. Crick's narrow formulation of the "central dogma" of molecular biology is now 60 years old and, unlike Watson's sloppier diagrammatic exposition, has not been overturned by subsequent results \cite{Morange2008a}. The fundamental idea of informational flow (defined in terms of primary sequence) from DNA to RNA to protein remains at the core of our theories of cellular function. The most serviceable accounts of the origin of life, beginning with autocatalytic nucleic acid reactions, are derived from the observed direction of macromolecular informational flow and not from any Darwinian evolutionary or ecological account \cite{Kauffman2000}. We still speak, in at least a loose sense, of genetic "programs" and developmental "trajectories". Theodore Dobzhansky's famous dictum that "nothing in biology makes sense except in the light of evolution" \cite{Dobzhansky1973}, offered at a time of growing anxiety among evolutionary biologists about the increasing dominance of the molecular program\footnote{Dobzhansky, in particular, was a theist, concerned to maintain the sense of a purposeful, directed, God-created universe in biological thought against what appeared to be a molecular reductionism that would eliminate it. Ironically, teleological language is now used far more freely in the molecular tradition than in the evolutionary one.}, is now obviously incorrect- it is, after all, the prosaic results of molecular methods that have elucidated the physical realities that evolutionary theory intends to describe, often in ways utterly unforseen by evolutionary biologists \cite{Shapiro2016}, and teachers are now urged to promulgate the Crickian dogma to explain evolution to their students \cite{Kalinowski2010}.

The success of the molecular program goes beyond its ubiquity, of course- it is adequate to explain, predict, and allow the manipulation of any number of biological phenomena, particularly at, but not limited to, the level of subcellular organisation. It is therefore worth questioning why it seems to have been eclipsed by the rise of "system biology", and indeed why only part of this thesis will offer recognisably "molecular" explanations for its objects of study. One may reasonably suspect that no real break has occurred here- on this view, "systems biology" merely refers to a evolution of the molecular program and the use of the term is a meaningless institutional fad \cite{Morange2008}. While it seems true that "[i]n bringing together strands of research that have little in common, the study of complexity sometimes seems to mean anything new and incomprehensible to nonspecialists that comes from the world of physics" \cite{Morange2003}, I suggest that the pervasive adoption of "complexity" rhetoric is more than empty intellectual cant, and instead reflects a collection of genuine, if divergent, efforts by biologists to address the epistemic and metaphysical limits of the molecular approach.

\section {Research Programs in Historical Context}

\subsection {Debunking of Scientistic Fairytales}

\subsection {Actually Existing Science, Limits in Principle and in Practice}

\subsection {Scientific Traditions as discursive communities within AES}

\subsection {}

Thomas Kuhn's observation

\section{Anomalies and irresolvables}

\subsection{Breakdown of the bioinformation paradigm}


This last point, that molecular biology is laden with explanatory metaphors and cognitive frames that are formally illegitimate, is not a trivial one. The concepts of mechanism and trajectory, for instance, have been imported from deterministic Newtonian dynamics. We may describe a protein as a machine, or describe the mechanism which gives rise to some cellular behaviour, which implies a similarly deterministic system. When pressed, however, we readily concede that this is merely a descriptive conceit; it is unnecessary to specify where a lathe is to know with a good deal of precision what it does, while the subcellular context of a protein can radically alter its functional properties, for instance. Likewise, a trajectory refers to the path a particle takes in space, which can be known precisely by specifying an initial condition and the time-reversible equations of motion it is subject to. We use this term only in the loosest sense when we speak of a cell following a developmental trajectory through some differentiation "phase space"- this process is generally neither subject to precise prediction given initial conditions, nor is it reversible in time. The use of these conceptual threads, mainly drawn from the epistemological framework of 19th century physics and 20th century information theory, while obviously not literal, has an effect on the resulting fabric of the description woven from their use. The widespread recognition that these descriptions can elide or even conceal much of the complexity and diversity of biological phenomena, often arising from apparently stochastic or chaotic processes, has, in part, given rise to the accusation that the molecular biological paradigm is a reductionist one.

This accusation is not completely unwarranted. The success of this paradigm gave rise to the hubris of the pregenomic era, when many were convinced that full genome sequencing would be sufficient to totally explain the organism supposedly specified by the information contained therein. This idea only makes sense in the context of a monomaniacal focus on a supposed informational "program" specified in nucleic acids being rigidly executed by proteinaceous "machines". Though this was never a commitment shared broadly by biologists, its obvious and public failure in the early part of this century has cemented the sense that molecular biology is inadequate in and of itself to explain the full range and diversity of biological phenomena. We are thus left with the ironic situation that the most successful approach to biological science in history is now regarded as being, in significant ways, fundamentally wrong about its subject; meanwhile exhortations to "holistic thinking" about biological systems offer little in the way of thoroughgoing alternatives or replacements to this approach, and tend to minimise or ignore the fraught history of previous "holistic" theories mentioned above.

\subsubsection{Recognition of biocomplexity \& proliferation of heterdox nondarwinian explanations for biological function}
\subsubsection{Paradigm-apotheotic molecular methods unable to produce predictable results (genetic compensation)}
\subsubsection{Failure of applied genetics to apply experimental methods in practical contexts}

\subsection{Stochasticity in Genetic Expression and Cellular Behaviour}

The challenge of offering a naturalistic account of the production of biological order and complexity from the mass of inert stellar ejecta comprising its material substratum is, arguably, the preeminent outstanding task of the biological sciences. While closely allied with questions of the origin and nature of life itself, these seemingly more fundamental issues have been addressed more satisfactorily from a materialist perspective (e.g. Kauffman's account of biological origins and Morange's tripartite definition of life) than the conundrum of how the rich diversity of terrestrial life can be produced from similar complements of macromolecules. The difficulty of this task is no more apparent than when one takes stock of the history of attempts to undertake it. The wreckage of vitalist theory, exemplified by Hans Driech's appeals to the Aristotlean metaphysical concept of entelchy (made after he abandoned his pioneering embryological studies on sea urchins), smoulders on as a warning to others who would offer nonreductionist accounts of the development of biological order- indeed, "vitalism" is still a serious charge levied at theorists today. By contrast, the mute edifice of Nobel laureate Ilya Prigogine's attempts to offer a thermodynamic explanation for the persistence of molecular biological order in the face of the seemingly inexorable workings of the Second Law (as "dissipative structures") is rarely remembered at all. No less a mind than Erwin Schroedinger considered the production of such a thermodynamic account critically important, going as far as to introduce the spurious concept of "negentropy" to do so. Almost this entire research program, once the focus of dedicated institutes, has been distilled to the (now) apparently trivial observation that organisms exist in open thermodynamic exchange with their environments, importing nutritive energy and exporting entropy in the form of heat. That some of the greatest scientific minds of the 20th century should have misstated the problem of biological order to this extent, such that uncounted careers were spent in pursuit of truths readily available to any farmhand observing the condensing breath of foraging cows on a cold morning, should give us pause.

\bigskip

Among the most pressing issues for stem cell biologists militating for the application of new mathematical and theoretical tools has been the observation of apparently stochastic proliferation and differentiation behaviour in stem and progenitor cells.

\subsubsection{Avoiding pathological interpretations of "stochastic" and "random"}

\subsubsection{Sources of stochasticity in cellular phenomena}


\subsection{The persistence of teleological frames}

Although it is now a shibboleth among evolutionary biologists that biological systems cannot be said to be directed toward ends in any meaningful sense, arising as they do from random variation subjected to natural selection, they are unable to offer any material account of this process without referring to the work of molecular biologists who casually refer to their subject in explicitly teleological terms (proteins have functions, cells are agents that pursue directed behaviours, etc.)
 

\section{Explanatory approaches to biological complexity}
\subsection{Choice of approach and model}
\subsubsection{Dangers of arbitrary model fits}
\subsubsection{Description is not understanding}

\begin{longquote}
The argument that enumeration of all of the molecular processes does not
yield an “understanding” of the system can be made also in the context of
computer simulations. To see this, let us suppose that we carried out an enormous molecular dynamics simulation incorporating the equations of motion
for all the molecules in some system and in this way we were able to track the
trajectory of each molecule. For example, assume that this was done for a gas
or liquid system and in this way we were able to precisely mimic the behavior
of an actual system at the molecular level. However, such a description would
not provide us with an understanding of this behavior. In particular, without
a knowledge of thermodynamics, the concept of entropy would not be deduced. Only after we have become aware of the concepts of thermodynamics
and we have attempted to imagine how these concepts are connected to microscopic behavior does it become possible to extract useful information from
such a simulation. Presently, there are plans for several projects to carry out
large-scale simulations employing models that incorporate all of the individual
processes taking place in a cell. However, as made clear from the preceding
discussion, even if such a project were successful and a precise description
accounting for all of the molecular-level processes taking place in a cell were
realized, this would not provide us with an understanding of living systems.
\end{longquote}
\cite{Kaneko2006}

\subsection{-omics enumerative projects}
\subsubsection{Enumeration and the explanatory Failure of -omics}
\subsection{Nonlinear dynamics}
\subsubsection{Nonlinear dynamic models as statistical and thermodynamic constructs}
\subsubsection{Nonlinear dynamic models applied to biological systes}
\subsubsection{Monte Carlo NDM solutions}
\subsection{Agent models}
\subsubsection{Emergent phenomena}
\subsubsection{Emergence and indeterminacy}
\subsubsection{What do agents "know"? The biosemiotic frontier}
\subsection{Constructive/Synthetic Biology}
\subsubsection{Why to prefer Sakai's engineering to spreadsheet jockeying}

\section{Suggested rubric for evaluating systems biological explanations}
\subsection{Deciding what kinds of explanations are valuable}
\subsection{The limits to scientific practice: achieving scientific sustainability}