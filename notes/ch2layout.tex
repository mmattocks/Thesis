\section{Development of "internal stochastic explanation" for retinal formation}

-Pam's "toward a unified thoery etc" paper

-

\section{"Internal stochastic explanation" is a pathological hypothesis arising at a time of recognized but unspeakable Kuhnian crisis for the biological sciences}

\subsection{Relevance of the claims}

-Hypothesis is not academic speculation, has both practical and fundamental import. We may divide

\subsubsection{What is the import of intrinsic, internal RPC dynamics?}

-serious claims with general relevance to stem cell function if tissue generation is in many ways specified by the development of robust internal dynamics within stem cells
-therapeutic implication: with the important consequence that producing the appropriate number and type of progeny is not affected by explantation, so two important guarantors of retinal functional properties (number and lineage distribution of cells) are stably safeguarded by stable properties of RPCs themselves.

\subsubsection{What is the import of claims about the stochasticity of biological systems?}

-Somewhat unclear: stochastic descriptions are statistical descriptions of \textit{deterministically} interacting classical dynamic systems whose evolution through time is either practically (measurement imprecision) or in principle (complexity/computational considerations) impossible to qualitatively predict by solving the relevant equation set. This is why Monte Carlo is used as a computational method to approximate the output of such a system.

-we are likely here more interested in the source of variability than in claims about whether variability exists. Boije et al. \cite{Boije2015} is a good example of an attempt to specify an underlying physical dynamic which gives rise to the source of variability, while gesturing to a theoretically-acceptable target for selection pressure which would allow this to be a evolved trait

-language of choice purports to recognise agency of cell while actually enslaving it to a randomly fluctuating oscillator. probably possible to recapitulate model output with a 555 IC on a breadboard with a few other detministically-interacting physical components.


\subsection{"Systems" explanations for sources of variability}

\subsection{Are "systems" claims $\Phi$-scientific or $\Sigma$-scientific? How do we compare them?}

\begin{longquote}
More precisely, in philosophy of science, Rein Vihalemm (2007) has pointed to a
different role that the historical explanation has in the major types of sciences, and on
this bases he has made a distinction  between $\Phi$-sciences and non-$\Phi$-sciences, the latter  called also $\Sigma$-sciences. 
$\Phi$-sciences do not require historical explanation, they
model the world using universal laws  
and depend on quantitative methods; $\Sigma$-sciences, instead, are dependent on historical  explanations, they model the world on qualitative basis and use primarily qualitative methods. 

The two are based on (and interrelated via) the semiosic practice, which includes
everyday  communication  and  practical  processes  of  classification  and  measurement
(Table 1).11
 As mentioned by Peirce, “measurement ... is a business fundamentally of
the same nature as classification” (CP  1.275). At the stage of modelling and theory,
however, the approaches diverge, $\Phi$-science as a modelling based on quantitative
convertibility, and $\Sigma$-science as a modelling based on qualitative diversity.  An
important  point  here  is  that  this  is  not  the  well-known  separation  between  the
humanities  and  the  sciences;  this  is,  instead,  science  as  it  necessarily  includes  the
complementary  approaches. 
in  principle,  $\Phi$-sciences can cover the whole world via
physical  descriptions, and $\Sigma$-sciences  can  cover  the  whole  world  as  the  sum  of
knowledge.   However, it would   be   weird for $\Phi$-sciences to describe the meaningfulness,  whereas
$\Sigma$-sciences  may  include  at  least  the  science  of  all  forms  of
life as much as it studies organisms’ knowledge and sign processes. 
\end{longquote}\cite{Kull2009a}

-methods of both types can provide "explanations" of variability ie: models that produce close fits with data

-we can legitimately translate between different kinds of explanations

-What Harris and Cayouette characterise as a "linear, deterministic" view of RPC development has strong $\Sigma$-scientific content (the previous states of cells matter- they have "historical continuity", and the phenomenon to be explained is the "ordered arrangement of qualitative diversity" that gives the retina its functional properties).

-The claim about "dynamic/stochastic \& linear/deterministic" may be that this $\Sigma$-scientific explanation for a system has failed because of its failure to include stochastic terms in its presumptions about variability in the system, but this is plainly untrue- H\&C concede that even the barest strawperson of a linear, deterministic description of fate specification has to include some distribution of cells across the "differentiation phase space". It is clear from early embryonic fate specification studies that the underlying physical dynamics of this system give rise to distributions of lineage production that would probably be well described by statistical methods (that is, the outcomes are distributed approximately normally in time, etc).

-That is, it has \textit{always been known} that lineage commitment involved "stochastic" variable phenomena, since the original clonal analyses in vivo.

\subsection{Structure of arguments}

-Since H\&C repeat the terms of these hypotheses in different forms over many publications, we can work out the logical structure of the model purported to explain zebrafish eye formation.

-giving one kind of explanation or another that produces a "good fit" to some set of data (particularly if it is normally distributed) may be a reason to suspect that the causal structure of the model corresponds with important causal dynamics of the phenomenon. 

-Perhaps more significant than the presence of variability ("stochastic" outcomes) is the imputed causal structure of the phenomenon described as "random" or "stochastic". 

-We know some of the sources of biological variability- being able to ascribe both tissue-level order and diversity in terms of noise in a relatively simple macromolecular "control circuit" would have obvious fundamental and practical applications.

-We can more usefully describe past hypotheses in terms of what processes they understand to be variable and what properties they assume about their objects of study. Which traits are assumed to be "linear" and "deterministic"? Which are expected to vary stochastically and with what results for the logic of the "control circuit" that must produce the retina? 

-Any "Timed Competency State" model could easily also have a nonlinear dynamic basis, so the difference in explantory value is what physical system these dynamics emerge from, and what "parameter space" is selected to describe the "differentiation landscape".

-H\&C are interpreting interactions between levels of transient, irreducible transcription/translation noise of two genes, as a selection-tunable nonlinear dynamic system. This is done to suggest a (neoDarwinian-tractable) physical basis for fate specification, which conveniently implicates alleles already under study. It is unclear that the model actually resembles this argument. Furthermore one could easily offer a "Timed Competency" version of this argument without any additional propositions:


\subsection{Harris et al. model analyses}

-Methods imported from $\Phi$-scientific contexts may or may not translate appropriately to $\Sigma$-scientific contexts if we accept the implications of Harris \& Cayouette's evidence. 

-The manner in which the Monte Carlo sim is conducted matters defines what it implies about the ostensible underlying physical structure of the modelled system.

-Is eg. the \cite{Boije2015} model actually "nonlinear" or "chaotic"? If not, the model does not point to the phenomenon it is supposed to describe- nonlinear dynamical interactions between two gene products. Instead, it is an arbitrarily defined oscillator (a biological die roll). The "rules" which interpret the outcome of die roll are specified by arbitrary selected external model constraints, sometimes drawn from model data, sometimes not.

-If H\&C are engaged in a cynical project (ends lay mainly outside the production of accurate, useful models \& publications), or badly guided by advice from outside their fields of competency (eg: individual nonbiologist personnel responsible for model formulation and testing), we may expect substantial divergence of the propositional structure of the models and that of the hypotheses ostensibly being tested by them.



\subsection{What is the import of claims about the stochasticity of biological systems?}

