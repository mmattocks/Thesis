-develop ch 2 historical background in a counterinductive context- how are actors trying to proceed and against what traditional backgrounds


\section{Development of fish retinal stem cell theory and models}

The first section of this chapter comprises a study of past efforts to theorise about and model the activity of retinal stem cells in the zebrafish (\textit{Danio rerio}), with a particular emphasis on the work of William A. Harris, a distinguished Canadian neurobiologist at the University of Cambridge.

\subsection{Theory/model distinction}

It is important to distinguish between theories and the models which they produce. A theory is the product of an interpretative reduction of some phenomenon to theoriser-salient semantic elements. Theories are often communicable in relatively plain language (technical jargon aside). A model is the product of a \textit{further} semiotic reduction, logically formalising the theory in terms of a system of (often mathematically described) rules. With this logical distinction, we may, for instance, accept a theory while disputing that a particular model is a good formal parameterisation of the plain language description.

\subsection{Theoretical background}

Empirical investigation of the cellular populations responsible for the formation, growth, and maintenance of fish retinas has been ongoing for over six decades, and has witnessed a number of dramatic transformations in the underlying theoretical framework used by biologists to understand their diverse subjects. As the earliest observation of mitoses in fish retinas was made in 1954 \cite{Vilter}, even the most basic conceptions about proliferative dynamics in the fish retina have been developing through most of the intellectual upheaval of 20th century biological theory.

For the purposes of this study, this diverse intellectual effort is simplistically divided into three broad periods:

\begin{enumerate}
\item Received Cell Biological Paradigm (1950s-1970s)
\item Molecular Mechanistic Information Paradigm (1970s-2000s)
\item "Systems" Paradigm Efforts (2000s-present)

\subsubsection{Cell Biological Paradigm (1950s-1970s)}

\subsubsection{Molecular Information Paradigm (1970s-2000s)}

\subsubsection{"Systems" Paradigm Efforts}

\subsection{Interpreting Harris' "Instrinsic Stochasticity Theory"}

\subsection{Model Formalisations of the Intrinsic Stochasticity Theory}
\subsubsection{Das 2003}
\subsubsection{Gomes 2011}
\subsubsection{He 2012}
\subsubsection{Boije 2015}

\subsection{Evaluation of IST and Its Models}
\subsubsection{Defects of Reality to Theory Translation}
\subsubsection{Defects of Theory to Model Translation}
\subsubsection{Model Utility}

\subsection{Empirical refutation of claims made for IST}


Selecting from among a diversity of possible models or interpretative frames requires reference to some criterion. This criterion (or set of criteria) is inevitably related to the focus of the investigator's concern. In the context of publicly funded science, a responsible investigator must organise their investigations with respect to their concern for the polity supporting this work. Purely "curiousity-driven research", divorced from public purpose, is only coincidentally "responsible" in this sense.

As grant applications are typically written before projects are fully organised, clearly articulated prospective connections between expected modelling output of a project and some vision of public interest or Good are rare. This produces a disconnect between the ostensible function of the granting process (directing public funds to proposals likely to advance the public interest) and the product of the scientific work funded thereby. This breach results in the admission of other structuring principles, so that, for instance, an investigator may be pressured to select an inadequate or counterproductive interpretative frame for their data in order to publish it before the next grant is applied for. The dynamic which results can stymie the formation of effective research programs, as valuable scientific resources are squandered developing and debating models selected for reasons unrelated to the advancement of program goals.
  
In order to determine what characteristics make a modelling approach more or less desireable, we must recognise that at the highest level, model design is arbitrary- a modeller is unconstrained in their selection of items included in the model, and in what will be abstracted away. As every model involves abstraction, there is no "true" model in the sense of a model which completely recapitulates the reality of some phenomenon. There are rather a plurality of useful, if incomplete, theoretical "viewpoints" to take up on phenomena which may be made to serve a variety of ends. 