\chapter{GMC\_NS.jl}

\path{GMC_NS.jl} implements a basic Galilean Monte Carlo nested sampler \cite{Skilling2012} in pure Julia. It is not capable of multiprocess likelihood calculations like \path{BioMotifInference.jl} and is not intended for high-dimensional problems requiring dynamic ensemble size adjustment \cite{Higson2019} or multi-ellipsoidal sampling \cite{Feroz2009}. \path{GMC_NS.jl} accepts arbitrary user-defined functions for model calculations, so could be used for any moderately-sized sampling problem with a continuous parameter space.

\section{Implementation notes}

The basic sampling scheme followed is, first, to initialize an ensemble of models with parameters sampled evenly from across the prior, then to compress the ensemble to the posterior by applying GMC moves to the least-likely model in the ensemble at a given iterate, constrained by its current likelihood (which is the ensemble's likelihood contour for that iterate). GMC tends to move model-particles across the parameter space very efficiently, particularly in the initial portion of compression across the uninformative bulk of the prior mass. That said, GMC particles may get "stuck" in widely separated minima, so there are instances when the least-likely particle has no valid GMC moves. In this case, the trajectory is not continued. Instead, a new trajectory is begun by resampling from the remaining prior mass within the ensemble. This is accomplished by one of two means. Firstly, an attempt is made to resample by "diffusion". This means a random particle remaining within the ensemble is selected, and random velocity vectors are sampled from this particle to find a starting location for the new trajectory (this vector is conferred to the new particle). If diffusional sampling fails (extraordinarily rare), a primitive ellipsoidal sampler serves as a backup. This draws an ellispoid around the ensemble's current models in parameter space and samples evenly from the ellipsoid. While this is an acceptable method of sampling evenly from within the existing likelihood constraint, it is wildly inefficient for posteriors with any kind of interesting topology, which is both why multi-ellipsoidal sampling exists \cite{Feroz2009}, and why this single-ellipsoid sampler is confined to a backup role.

\path{GMC_NS.jl} follows the common convention of transforming parameter space to a unit-standardized hypersphere. The user is responsible for ensuring that the density of the prior is uniform, that is $\pi(x) = 1$, over the -1:+1 range of the hypersphere dimension. Utility functions \path{to_unit_ball} and \path{to_prior} which accept the prior and transformed positions and the relevant prior probability distribution, returning the unit hypersphere or prior coordinate, respectively. In future versions, this scheme is likely to be modified to a 0:1 unit hypercube, which does not require additional operations beyond obtaining the cumulative probability of a parameter value on the prior distribution. The user may also specify a sampling "box" to bound particles from illegal or nonsensical areas of the parameter space (eg. negative values for continuous distributions on positive-valued physical variables). The box reflects particles specularly in the same manner as the likelihood boundary, with the exception that because the orientation of the nth-dimensional box "side" is known (eg. is the plane at n=0.), the reflection is performed by stopping the particle at the box side and giving it a new velocity vector identical to the old but with reversed sign in dimension n. 

\path{GMC_NS.jl} attempts to maintain efficient GMC sampling by per-particle PID tuning of the GMC timestep. GMC treats models as particles defined by their parameter vectors, which give their positions in parameter space, as well as a velocity vector, normally chosen isotropically and only changed when the particle "reflects" off the ensemble's likelihood contour. In GMC, when a model-particle is to be moved through the parameter space (in this case, because it is the least likely particle in the ensemble), a new position is proposed by extending some distance along its velocity vector through the parameter space. This distance is determined by scaling the velocity vector by a "timestep" value, which can be thought of as the speed with which the particle is covering parameter space. As the model ensemble is compressed down to the posterior, this timestep must decline fairly evenly in order for GMC to be efficient, as the amount of available parameter space declines rapidly. Additionally, GMC particles may enter convoluted regions of parameter space that form small likelihood-isthmuses to other, more likely regions. In order to deal with this, each trajectory is assigned its own PID tuner, which maintains an independent timestep for the trajectory, tuned to target a user-supplied unobstructed move rate. In short, this will reduce the timestep if the particle is repeatedly reflecting off the likelihood boundary (in which case it is crossing the remaining prior mass without sampling much from it, or it is in a highly convoluted area of the local likelihood surface), and extend the timestep if it repeatedly moves without encountering the boundary, so that particles that are closely sampling an open region without encountering the boundary begin to "speed up". This scheme tends to produce a "shell" of swirling GMC particles reflecting obliquely off the likelihood contour, which gives good linear sampling of the entire likelihood range.

\section{Ensemble, Model, and Model Record interfaces}

\section{Usage notes}