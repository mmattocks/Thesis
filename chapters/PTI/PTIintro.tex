\section*{Pr\'{e}cis of Modelling Studies}
\label{ch:precis}
\addcontentsline{toc}{chapter}{\nameref{ch:precis}}

The primary concern of this thesis is developing and assessing computational models of retinal progenitor cell (RPC) phenomena. While many different, well substantiated explanations for various aspects of eye development exist, these explanations have usually been supported by assessing the frequentist significance of observed effects, not by building formally testable models of the phenomena themselves. Of the explanations which could be subject to model comparison, virtually all of these are derivatives of the Simple Stochastic Model (SSM), dating to the earliest work, performed by Till and McCulloch \cite{Till1964}.

The explosion of molecular biological information has massively increased the number of parameters that might be included in RPC models. The question of how RPC activity might be successfully predicted and controlled in vivo may revolve around finding adequate models that explain retinogenetic processes. The development and assessment of such models is, therefore, of fundamental basic and applied interest. This is particularly so, in light of increasing interest in entraining RPCs for the purpose of retinal regenerative medicine. Our group is particularly interested in the possibility that RPCs located in the peripheral retinal annulus of the circumferential marginal zone (CMZ), which are present in zebrafish and mammals alike, could be harnessed for retinal repair.

It is surprising that, despite the use of SSMs, the literature on neural stem cells contains virtually no systematic statistical approaches to the construction, optimization, or comparison of models. This thesis is an attempt to address this problem. It proceeds in three basic strokes. Firstly, Chapters \ref{chap:RPCreview} and \ref{chap:SMME} evaluate the existing explanations of RPC behaviour, and find that those with formal model components are deficient, and cannot explain the activities of RPCs in the CMZ. Secondly, Chapters \ref{chap:SMMEoutro} and \ref{chap:CMZ} propose a general CMZ modelling framework under which the Bayesian evidence for different model-hypotheses might be assessed, and, by testing a variety of population-level hypotheses, provide guidance and develop the methods required to test cell-based hypotheses. Lastly, Chapters \ref{chap:rysintro}, \ref{chap:rys}, and \ref{chap:rysoutro} introduce the zebrafish CMZ mutant \textit{rys}, and apply some of the methods developed in the second stroke to explain aspects of the aberrant morphology and behaviour of \textit{rys} RPCs. 

\autoref{chap:RPCreview} begins by summarizing the range of explanations that have been offered for RPC activities, and introduces the primary set of formal models that have recently been used to favour an explanation centered around a supposed stochastic mitotic mode of RPCs. \autoref{chap:SMME} elucidates the structure and development of the stochastic mitotic mode explanation (SMME) models, finding fundamental logical errors in their interpretation of "stochasticity". Moreover, by pursuing a basic information theoretical model selection approach, it is demonstrated that the SMME is not even the best available explanation relying on this flawed notion of stochasticity.

\autoref{chap:SMMEoutro} discusses the implications of \autoref{chap:SMME} for models of RPCs generally, and in the CMZ particularly; it also recommends a better model selection approach from Bayesian theory, nested sampling. \autoref{chap:CMZ} surveys the activity of RPCs in the post-embryonic zebrafish CMZ, developing Bayesian methods for doing so. These methods include the use of nested sampling to solve the long-standing problem of estimating cell cycle parameters of subpopulations of cells from cumulative thymidine labelling measurements of the entire super-population. It concludes with recommendations for future modelling approaches to the CMZ, structured by the foregoing Bayesian analysis. 

 \autoref{chap:rys} identifies the causative mutation in \textit{rys} and shows that the zebrafish CMZ mutant \textit{rys} CMZ RPCs have disorganized chromatin, likely blocking differentiation but not proliferation. Nested sampling is used to demonstrate that the nuclear phenotype is likely the consequence of a shift in nucleosome histone composition in \textit{rys} progenitors. \autoref{chap:rysoutro} discusses some of the theoretical issues raised by the \textit{rys} analysis, and their implications for modelling retinogenesis more broadly, especially in integrating nuclear dynamics into models of RPCs.