\chapter{Inferring and modelling specification dynamics in retinal progenitors}
\label{chap:rysoutro}

\section{Specification as a function of non-proliferative chromatin dynamics}
The npat mutant \textit{rys} investigated in \autoref{chap:rys} reinforces a basic theme running through this thesis, that lineage commitment or specification can be, and often is, dissociable from the proliferative behaviour of retinal progenitors, as has been suggested by Engerer et al. \cite{Engerer2017}. \textit{rys} illustrates this: RPCs undergo a final burst of proliferation before the death of the animal\footnote{Likely, on this view, arising from systemic failure of progenitors to specify appropriately.}, but never contribute appropriately to the neural retina. Indeed, older \textit{rys} larvae display a proliferative spillage of abberant RPCs into the intraocular space between the lens and GCL; the progeny of these abberant RPCs literally cannot be integrated into the tissue.

The importance of this dissociability for our inferences about RPCs can be gleaned from the basic problems encountered in each of the modelling studies. To begin with, the Stochastic Mitotic Mode Explanation models examined in \autoref{chap:SMME} failed, in large part, because the underlying Simple Stochastic Models conflate proliferation and specification under the ``mitotic mode'' concept. This leads immediately to the introduction of model structure to explain specification outside of ``mitotic mode'', assuming a phased temporal structure to RPC behaviour, without attempting to explain it. Lineage specification and niche exit processes need to be modelled separately from mitotic processes to differentiate their effects on outcomes. Next, proliferative data and morphological measurements in \autoref{chap:CMZ} fail to constrain the posterior distributions on the rate of niche exit in simple phased models, demonstrating that this parameter cannot be inferred with any certainty from overall population-level proliferative dynamics. To accurately model the behaviour of RPCs, data which provides accurate volumetric or temporal constraints on modelled specification or niche exit rate is of paramount importance. Finally, in \textit{rys}, we encounter the problem of explaining RPCs that proliferate but fail to specify; we explain this phenomenon in terms of the effects of mutant npat on chromatin organization by way of effects on histone expression. This directs us to the relevant level of organization for explanation of RPC lineage outcomes: the progression of chromatin states that underlay long-term changes in gene expression, which are required for the generation of the specified protein complement. Our  approach, while providing evidence for the chromatin dynamics we suggest arise from the effects of npat on the available histone pool, does not give us insight into how the appropriate progression of chromatin states might be usefully parameterised.

While a large number of techniques for studying aspects of chromatin structure exist, few suggest a means to describe an overall ``nuclear state''. That is, if we consider Waddington's idea of a cell's lineage state being a ball rolling down a canalized terrain \cite{Waddington1957}, it remains unclear what sort of parameter space would allow us to infer where the ball is. Older ideas of cellular state space, that rely exclusively on measures of protein expression, are unable to address hypotheses about chromatin structure. It remains unclear what aspects of chromatin organisation are causally important in fate commitment, as little formal model comparison has taken place. One might propose a joint structure-sequence-expression space with thousands of variables drawn from the vast literature on chromatin organisation. Such a project is unlikely to succeed with reasonable compute budgets, even with the sophisticated Bayesian inference tools developed here.

Still, if it is not obvious how one might build a general model of the evolution of ``nuclear state'', we may suggest some more limited approaches arising from the work presented in this thesis, that may make some headway here.

\section{Future directions}
\textit{rys} provides a useful platform for investigating the processes required to organize the progression of chromatin through the sequence of states we presume to be required for progression through the series of stages implied by the studies presented in \autoref{chap:SMME} and \autoref{chap:CMZ}, and the associated shifts in RPC lineage contributions. Despite the evidence amassed for the involvement of npat effects on nucleosome positioning in \textit{rys} mutants, many aspects of the suggested mechanism remain obscure. This arises, in part, from the lack of reliable macromolecular tools for investigating npat and histone proteins in \textit{D. rerio}. It would be useful to focus inquiry here on a small number of candidate histones. However, the large number of zebrafish histone genes, with no crystallographic data available, suggests that prediction of the involvement of particular histones would be difficult.

This situation is now rapidly changing, however. The AlphaFold algorithm has recently been recognized as a solution to the Critical Assessment of protein Structure Prediction (CASP) problem set in the CASP13 competition \cite{AlQuraishi2019}. This is the first time protein folding has been adequately predicted \textit{in silico} from sequence data alone. If it is possible to use this algorithm to predict \textit{D. rerio} nucleosome structures from histone folding predictions, this may obviate the need for crystallographic data. Such structural predictions could be used with the posterior estimates of signals putatively arising from nucleosome-DNA contact produced in \autoref{rysmotifs} to, in turn, predict which nucleosome compositions are most likely to favour the detected sequences. This might be assisted by 3d pattern matching for DNA sequences \cite{Herisson2007}, or roll-and-slide modelling of the contact motifs \cite{Tolstorukov2007} Such a study would produce hypotheses that, if interesting enough to pursue \textit{in vivo}, would supply estimates of how likely various histones are to be involved in the \textit{rys} mutant phenotype, and in what regard, which would help prioritise resources for the development of assays for interrogating the specific transcript and protein status of the most likely loci.

A more fundamental way of interrogating \textit{rys} ``nuclear state'' might be provided by focusing on transcription itself, which has been proposed as a primary driver of chromatin organisation \cite{Cook2018}. Such a project would likely require a transgenic line developed to monitor the transcription of one or more loci of interest in live cells, perhaps using bio-orthogonally labelled transcripts, as has recently been demonstrated in zebrafish \cite{Westerich2020}. This approach could possibly be used either to investigate the status and trafficking of npat and histone transcripts themselves, or to determine the impact of the npat lesion on the transcriptional shaping of chromatin organisation by tracking the transcription of progenitor markers, for instance.

Ultimately, it is likely that models that allow the prediction and control of RPC behaviour \textit{in situ} in zebrafish retinae will treat the chromatin organisation of RPC nuclei as an integrator of multiple relevant sources of information, including physical forces, cytoskeletal dynamics, and extracellular signals. This type of functional organisation, in which biological ``meaning'' arises from the integration of salient cellular information by a combination of nonarbitrary physicochemical relations and arbitrary biological coding reltaions, is currently best described by the subdiscipline of biosemiotics \cite{Hoffmeyer2008,Favareau2015,Hoffmeyer2015}. Of particular interest here is the clear-eyed and unobscurantist work of Marcello Barbieri, which may well provide the metaphysical framework necessary to plumb the depths of this topic \cite{Barbieri2014}.