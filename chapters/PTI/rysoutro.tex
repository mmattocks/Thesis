\chapter{Inferring and modelling nuclear dynamics in retinal progenitors}
\label{chap:rysintro}

\section{Intro}
transition from discussion of modelling postembryonic CMZ and using nuclear shape to infer identity for modelling purposes to discussion of the implication of the correlations of nuclear state, cellular identity, and cellular function

\section{Rys}
introduce zebrafish eye mutants in general and rys in particular

\section{Histones and DNA sequence}
Background material on histone positioning and the effect of DNA sequence identity

\section{Inferential logic}
briefly outline the chain of inference made in the paper and the influence of technical limitations, explain calculating evidence ratios by nested sampling as a form of Bayesian model selection
\section{Technical caveats}
List important caveats to keep in mind when reading the paper

\section{Separability of cell cycle and specification}
practical import of this is high- compare the conflation of Harris models to possible model framework which separates and modulates these independently for eg. transplants
\section{Frontiers of nuclear dynamics}
discuss recent experiments measuring chromatin conformation, pulsatile transcription etc
\section{Integrating nuclear dynamics into models of CMZ- abstraction and biosemiotics}
abstraction necessary due to computational limits, biosemiotics allow for the calibration of this theoretically


