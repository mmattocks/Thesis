\section*{Introductory Notes}
\label{ch:intro}
\addcontentsline{toc}{chapter}{\nameref{ch:intro}}

\subsection*{Thesis Guide}
This document is split into three parts. Part I contains results and discussion pertaining to empirical modelling studies that will be of interest to committee members and developmental biologists. Part II contains technical reports pertaining to novel software that was written in order to perform the analyses presented in Part I. Part III contains a variety of supplementary materials arising from Parts I and II. These include detailed descriptions of the methods used in Part I, less-technical explanations of relevant statistical and model theory, the source code of all software and analyses, and the bibliography. The source code has been omitted from the print document.

Readers of the .pdf document will find some text unobtrusively highlighted in plum throughout the thesis. Section indices highlighted in this way link to the specified section. Technical terms have also been highlighted when pertinent material is available in Part III to explain them for the reader who may be unfamiliar.

\subsection*{Terminology and Style}
Throughout \autoref{chap:RPCreview}, \autoref{chap:SMME}, and \autoref{chap:SMMEoutro}, I have used the apellation "Harris" when referring to the output of William Harris' research group. This is a large body of research spanning several decades, and involves many co-authors. My use of "Harris" here is a convenience, as Harris is the only common author across the period in question, and presumably the agent carrying these ideas forward from project to project. It is not intended to slight or minimize the contributions of any of the other members of Harris' group (many of whom are now senior scientists in their own right). I have used "Raff" in an identical sense in referring to the Raff group's work in \autoref{chap:SMMEoutro}.

I have preferred the terms "specification", "determination", and their derivatives, to refer to cells assuming a particular lineage fate. "Differentiation" is well-understood, but ambiguous, and often understood to relate to the mitotic event itself, which I generally do not intend. "A cell has specified" means it has assumed a stable macromolecular identity or "fate". This term is intended to correspond exactly with the appearance of stable markers of cell type.

There are a number of specialised philosophical terms that appear in quoted material in the Part III. I have made occasional use of some use of them to save space. These are as follows:

iff - "if and only if"
explanadum - the fact to be explained
explanans - the explanation itself 

Unless otherwise noted, formatting of quoted material is preserved, so that italic emphases appear in the original. An exception to this is citations in original material, which have been removed wherever irrelevant or uninformative (ie. numbered references). I have indicated my own editorial comments and alterations to quoted material with [square braces].